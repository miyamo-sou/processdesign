\documentclass[a4j]{jsarticle}
\usepackage[dvipdfmx]{graphicx}

\begin{document}

% 表紙
\title{トルエンの空気酸化による安息香酸の製造}
\author{荊尾大雅、宮本奏汰}
\maketitle

% 本文
\chapter*{第1章 緒言}
安息香酸は、主としてフェノールの原料となる他、その塩が食品や化粧品などの添加物として広く利用されている。
2014年には世界全体で 万トンが製造されており、新興国での需要から、2024年には生産量が 万トンとなると見込まれている。
そこで、私たちは原料として安価なトルエンを使用し、空気酸化することで安息香酸を製造するプロセスを
設計し、本プロセス設計演習において検討することにした。

\chapter*{第2章 プロセスの概要}
\section*{2.1 プロセスの概要}
本設計で対象とするのは、トルエンを空気酸化することにより安息香酸を製造するプロセスである。
プロセス全体の概略図を以下に示す。

\section*{2.2 設計条件}
生産要求は、99.0wt\%以上の安息香酸を年2万トンとする。\\
工場の稼働時間は、1日24時間、年300日とする。\\
原料として、純度100\%のトルエンおよび、組成を窒素79mol\%酸素21mol\%とする空気を用いる。
ただし、原料は25℃、1barで供給されるものとする。\\
減価償却期間は7年である。

\chapter*{第3章 反応部}
反応部の概略図を以下に示す。

\section*{3.1 反応機構}

\section*{3.2 反応器選定}
流通式の気液反応器の種類には、拡散に対して不利な順に、気泡塔、気液攪拌槽、充填塔などがある。
最大反応速度と最大拡散速度の比を表す八田数を事前に試算し、その値によって気液攪拌槽型反応器を選択した。
\begin{equation}
    \gamma = \frac{(最大反応速度)}{(最大拡散速度)} = \frac{\sqrt(C_{{\rm B}})}{A}
\end{equation}

\section*{3.3 設計方程式}
液相はCSTRとして、気相は鉛直方向のPFRとして設計を行った。
用いた仮定は以下のようになる。
\begin{itemize} 
    \item 気相は水平方向に一様な濃度分布を持つ。\\
    \item 気相側境膜抵抗は無視できる・\\
    \item 液相は完全混合状態である。\\
    \item 窒素、酸素はヘンリー則に従い、その他の物質はラウール則に従う。
\end{itemize}
以上の仮定および、蒸発油分を還流する機構を含めて、設計方程式を立式した。\\
\begin{equation}\nonumber
    0=F^{{\rm in}}_{{\rm liq},j}-F^{{\rm out}}_{{\rm liq},j} -(1-\beta_j) k_{{\rm L}}a
    \int^{V_{{\rm tot}}}_0(C_j - C^{{\rm sat}}_j){\rm d}V + r_j V_{{\rm L}}
\end{equation}
\begin{equation}\nonumber                                                
    \frac{{\rm d}F_{{\rm gas},j}}{{\rm d}V} = k_{{\rm L}}a(C_j - C^{{\rm sat}}_j)
\end{equation}

\section*{3.4 物質移動容量係数の推算}
物質移動容量係数の推算に用いた各相関式を記す。

液相側物質移動係数の相関式\\
小気泡の場合
\begin{equation}
    k_{{\rm L}} = 0.31Sc_{{\rm L}}^{-2/3}(g \Delta \rho \mu_{{\rm L}}/\rho_{{\rm L}}^2)^{1/3}
\end{equation}
大気泡の場合
\begin{equation}
    k_{{\rm L}} = 0.42Sc_{{\rm L}}^{-1/2}(g \Delta \rho \mu_{{\rm L}}/\rho_{{\rm L}}^2)^{1/3}
\end{equation}
比表面積の相関式
\begin{equation}
    a = 1.44(\frac{P_{{\rm V}}^{0.4} \rho_{{\rm L}}^{0.2} }{ \sigma^{0.6}})(\frac{u_{{\rm G}}}{u_{{\rm t}}})^{0.5}(\frac{P_{{\rm T}}}{P_{{\rm G}}})(\frac{\rho_{{\rm G}}}{\rho_{{\rm a}}})^{0.16}
\end{equation}

上記の2式を利用するために用いた相関式を以下に記す。

ガスホールドアップの相関式
\begin{equation}
    \varepsilon_{{\rm G}} = (\frac{u_{{\rm G}}\varepsilon_{{\rm G}}}{u_{{\rm t}}}) + 0.000216 \times(\frac{P_{{\rm V}}^{0.4} \rho_{{\rm L}}^{0.2} }{ \sigma^{0.6}})(\frac{u_{{\rm G}}}{u_{{\rm t}}})^{0.5}(\frac{P_{{\rm T}}}{P_{{\rm G}}})(\frac{\rho_{{\rm a}}}{\rho_{{\rm G}}})^{0.16}
\end{equation}
気泡の体積平均径の相関式
\begin{equation}
    d_{{\rm vs}} = 4.15 (\frac{\sigma^{0.6}}{P_{{\rm V}}^{0.4} \rho_{{\rm L}}^{0.2}})(\frac{P_{{\rm G}}}{P_{{\rm T}}})(\frac{\rho_{{\rm a}}}{\rho_{{\rm G}}})^{0.16} \varepsilon_{{\rm G}}^{0.5} + 0.0009
\end{equation}
気泡の終末速度の相関式
\begin{equation}
    u_{{\rm t}} = (\frac{4\Delta \rho g d_{{\rm vs}}}{3C_{{\rm D}}\rho_{{\rm L}}})^{0.5}
\end{equation}

さらに、上記の相関式を利用するために用いた物性値の推算式、および変数の定義式などの諸式を以下に記す。

抗力係数の相関式
\begin{equation}
    C_{{\rm D}} = {\rm max}[\frac{24}{Re}(1+0.15Re^{0.687}), \frac{8}{3}\frac{Eo}{Eo+4}]
\end{equation}
拡散係数の推算\\
wilke-changの式
\begin{equation}
    D_{12} = \frac{2.946\times 10^{-11}(\beta M_{{\rm r,2}})^{1/2} T} {\mu_2 V_{{\rm b,1}}^{0.6}}
\end{equation}
Einstin-Stokesの式
\begin{equation}
    \frac{D \mu}{T} = {\rm const}
\end{equation}

表面張力の推算\\
表面張力の温度依存性に関する相関式
\begin{equation}
    \sigma \propto \{1-(T/T_{{\rm c}}) \}^n    
\end{equation}

気泡レイノルズ数の定義式
\begin{equation}
    Re = \frac{\rho_{{\rm L}}u_{{\rm t}}d_{{\rm vs}}}{\mu_{{\rm L}}}
\end{equation}
エトベス数の定義式
\begin{equation}
    Eo = \frac{g \Delta \rho d_{{\rm vs}}^2}{\sigma}
\end{equation}

\section*{3.5 反応器設計結果} 
反応部の概略図および物質収支を以下に示す。

\chapter*{第4章 分離部1}
\section*{4.1 蒸留塔設計}
未反応トルエンのうち99\%以上を回収することを目的とした。
設計条件を以下に記す。

各蒸留塔圧力の設定値において、反応器体積および晶析体積を操作変数として評価関数の値を最大化し、
評価関数の最大値同士を比較することにより蒸留塔圧力を決定した。

\chapter*{第5章 分離部2}
\section*{5.1 晶析器設計}
\subsection*{5.1.1 晶析器選定}
溶解度の温度依存性が大きいことと、目的とする結晶生産量が大きいことから、
連続式攪拌槽型反応装置を選定した。
\subsection*{5.1.2 設計方程式}
以下の仮定を用いた。
\begin{itemize} 
    \item 結晶表面拡散は迅速に行われる。\\
    \item 晶析器内は完全混合状態である。\\
    \item 二次核発生の影響は無視する。\\
\end{itemize}

参考文献[]により、以下の実験式および理論式を用いて設計を行った。\\
一次核発生速度
\begin{equation}
    B^0 = k_{{\rm b}}M_{{\rm T}}^j \Delta C^b
\end{equation}
結晶成長速度
\begin{equation}
    G = k_{{\rm g}}\Delta C^g
\end{equation}
結晶成長速度定数
\begin{equation}
    k_{{\rm g}} = k_{{\rm g0}} \exp(-\frac{E_{{\rm g}}}{RT})
\end{equation}
個数収支式
\begin{equation}
    n=n^0 \exp(-\frac{L}{G\tau})
\end{equation}
懸濁密度
\begin{equation}
    M_{{\rm T}} = c_0-c = 6k_{{\rm v}}\rho_{{\rm c}}n^0(G\tau)^4
\end{equation}

\subsection*{5.1.3 晶析器設計結果}
晶析器体積と晶析器内温度を操作変数として、評価関数を最大化する2次元探索を行った。
以下に探索によって得た図を示す。
設計結果を以下に記す。

\section*{5.2 抽出塔設計}
十分に塔内へ液を滞留させることによって安息香酸を水中に飽和させることを目的とした。
設計仮定を以下に記す。
設計条件を以下に記す。

\chapter*{第6章 最適化}
\section*{6.1 方法}

\section*{6.2 蒸留塔圧力の最適化}

\section*{6.3 晶析器内温度の最適化}

\chapter*{第7章 物質収支 $\cdot$ 熱収支}
全体のフロー図および流量関係は下図のようになった。

全体の熱量関係は以下のようになった。

\chapter*{第8章 ヒートインテグレーション}
流体同士の熱交換を行い、外部流体の利用量を削減した。
以下に最終的な設計結果におけるTQ線図を示す。

熱交換により、外部熱媒を用いた場合と比較して、

\chapter*{第9章 経済評価}

\chapter*{第10章 結言}
設計目標を純度99.0wt\%の安息香酸を年2万ton製造するものとして、設計を行った。
文献値を参考として反応器、および晶析装置を設計した。
また、リサイクルフローおよび燃焼炉の設置によって原料を有効活用できた。
最適化については、晶析器内温度および蒸留塔内圧力を順次最適化し、
プロセス全体について、反応器体積と晶析器体積を用いて最適化することができた。
本設計では積極的に多変数を用いて最適化を行った。
これにより、年 億円の利益を得ると見込める設計となった。\\
課題としては、さらに多くの変数を用いた最適化、特に反応器内の条件を変更して行うこと、
また、燃焼炉ではなく、吸着装置を用いた時の比較検討を行うこと等が挙げられる。

\chapter*{謝辞}

\chapter*{参考文献}

\chapter*{Appendix}

\end{document}
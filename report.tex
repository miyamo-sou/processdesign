\documentclass[platex, a4j]{jsarticle}
\usepackage[dvipdfmx]{graphicx}

\begin{document}

% 表紙
\title{トルエンの空気酸化による安息香酸の製造}
\author{荊尾大雅 \and 宮本奏汰}
\maketitle

% 本文
\chapter*{第1章 緒言}

\chapter*{第2章 プロセスの概要}

\section*{2.1 設計目標}
\section*{2.2 プロセスフロー}


\chapter*{第3章 反応部}

\begin{figure}[h]
    \includepraphic[scale=1.0]{}
    \caption{反応部の概略図}
\end{figure}

\section*{3.1 反応機構}
\section*{3.2 反応器選定}
流通式の気液反応器の種類には,拡散に対して不利な順に,気泡塔,気液攪拌槽,充填塔などがある.
最大反応速度と最大拡散速度の比を表す八田数を事前に試算し,その値によって気液攪拌槽型反応器を選択した.
\begin{equation}
    \gamma = \frac{(最大反応速度)}{(最大拡散速度)} = \frac{\sqrt(C_{{\rm B}})}{A}
\end{equation}

\section*{3.3 設計方程式}
液相はCSTRとして,気相は鉛直方向のPFRとして設計を行った.
用いた仮定は以下のようになる.
\begin{itemize}
    \item 気相は水平方向に一様な濃度分布を持つ.\\
    \item 気相側境膜抵抗は無視できる・\\
    \item 液相は完全混合状態である.\\
    \item 窒素,酸素はヘンリー則に従い,その他の物質はラウール則に従う.
\end{itemize}
以上の仮定および,蒸発油分を還流する機構を含めて,設計方程式を立式した.\\
\begin{equation}\nonumber
    0=F^{{\rm in}}_{{\rm liq},j}-F^{{\rm out}}_{{\rm liq},j} -(1-\beta_j) k_{{\rm L}}a
    \int^{V_{{\rm tot}}}_0(C_j - C^{{\rm sat}}_j){\rm d}V + r_j V_{{\rm L}}
\end{equation}

\begin{equation}\nonumber
    \frac{{\rm d}F_{{\rm gas},j}}{{\rm d}V} = k_{{\rm L}}a(C_j - C^{{\rm sat}}_j)
\end{equation}

\section*{3.4 反応器設計結果}


\chapter*{第4章 分離部1}
\section*{4.1 蒸留塔設計}


\chapter*{第5章 分離部2}

\section*{5.1 晶析器設計}
\subsection*{5.1.1 晶析器選定}
溶解度の温度依存性が大きいことと,結晶生産量が大きいことから,
連続式攪拌槽型反応装置を選定した.
\subsection*{5.1.2 設計方程式}
以下の仮定を用いた.
\begin{itemize}
    \item 結晶表面拡散は迅速に行われる.\\
    \item 晶析器内は完全混合状態である.\\
    \item 二次核発生の影響は無視する.\\
\end{itemize}

参考文献[]により,以下の実験式および理論式を用いて設計を行った.\\
一次核発生速度
\begin{equation}
    B^0 = k_{{\rm b}}M_{{\rm T}}^j \Delta C^b
\end{equation}
結晶成長速度
\begin{equation}
    G = k_{{\rm g}}\Delta C^g
\end{equation}
結晶成長速度定数
\begin{equation}
    k_{{\rm g}} = k_{{\rm g0}} \exp(-\frac{E_{{\rm g}}}{RT})
\end{equation}
個数収支式
\begin{equation}
    n=n^0 \exp(-\frac{L}{G\tau})
\end{equation}
懸濁密度
\begin{equation}
    M_{{\rm T}} = c_0-c = 6k_{{\rm v}}\rho_{{\rm c}}n^0(G\tau)^4
\end{equation}

\subsection*{5.1.3 晶析器設計結果}


\section*{5.2 抽出塔設計}

\chapter*{第6章 最適化}
\section*{6.1 方法}
\section*{6.2 蒸留塔圧力の最適化}
\section*{6.3 晶析器内温度の最適化}

\chapter*{第7章 物質収支 $\cdot$ 熱収支}

\chapter*{第8章 ヒートインテグレーション}

\chapter*{第9章 経済評価}

\chapter*{第10章 結言}

\end{document}


\documentclass{jsarticle}
\begin{document}

\section{スライド用数式}

液相物質収支式
\begin{equation}\nonumber
        0=F^{{\rm in}}_{{\rm liq},j}-F^{{\rm out}}_{{\rm liq},j} -(1-\beta_j) k_{{\rm L}}a
        \int^{V_{{\rm tot}}}_0(C_j - C^{{\rm sat}}_j){\rm d}V + r_j V_{{\rm L}}
\end{equation}

気相物質収支式
\begin{equation}\nonumber                                                
        \frac{{\rm d}F_{{\rm gas},j}}{{\rm d}V} = k_{{\rm L}}a(C_j - C^{{\rm sat}}_j)
\end{equation}

収支式変数説明
\begin{alignat*}{2}
        &F_{j}&&:\text{成分$j$のモル流量}                       \,[{\rm mol\cdot s^{-1}}] \\
        &C_{j}&&:\text{成分$j$のモル濃度}                       \,[{\rm mol\cdot m^3}]\\
        &r_{j}&&:\text{成分$j$の反応速度}                       \,[{\rm mol \cdot m^{-3}\cdot s^{-1})}]\\
        &\beta_{j} &&:\text{成分$j$のデカンターでのパージ率}     \,[\,-\,] \\
        &k_{{\rm L}}a&&:\text{物質移動容量係数}                 \,[{\rm s^{-1}}] \\
        &V_{{\rm tot}} &&:\text{反応器内の液相部と気相部の合計体積}   \,[{\rm m^3}]\\
        &V_{{\rm L}}&&:\text{反応器内の液相部の体積}             \,[{\rm m^3}]\\
        &{{\rm sat}}&&:\text{飽和状態を表す添え字}         
\end{alignat*}                                                  

晶析方程式説明
\begin{alignat*}{2}
        &n^{0}&&:=\,B^0/G                                                \\
        &L&&:\text{結晶の特徴径}                         \,[{\rm m}]\\
        &\Delta C&&:\text{水中BzAの過飽和度}              \,[{\rm g/g}]\\
        &c &&:\text{水中BzAの濃度}                        \,[{\rm g/mL}] \\
        &\tau&&:\text{平均滞留時間}                       \,[{\rm min}] \\
        &k_{{\rm v}}&&:\text{体積形状係数}                \,[\,{\rm -}\,]\\
        &\rho_{{\rm c}} &&:\text{BzA結晶密度}             \,[\,{\rm g/cm^3}\,]\\
        &E_{{\rm g}}&&:\text{活性化エネルギー}             \,[{\rm kJ/mol}]\\   
        &k_{{\rm b}}&&:\text{核発生速度定数}             \,[(\#/({\rm m^3 \cdot s}))/({\rm g/ mL})^{\it j}]\\ 
        &k_{{\rm g0}}&&:\text{核成長頻度因子}             \,[{\rm \mu m/s}]\\    
        &b,\,g,\,j&&:\text{実験による定数}             
\end{alignat*}

\end{document}